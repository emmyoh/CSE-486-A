\lesson{Fri, 15 September 2023, 11:40am -- 1:00pm}{Week 3, Friday}

\subsection{$\alpha$-$\beta$ pruning -- continued}
\label{sub:alpha_beta_pruning_continued}

$\alpha$-$\beta$ pruning has the property of reducing the branching factor, $b$, to its square root (ie, $b \stackrel{\alpha \beta}{\rightarrow} \sqrt{b}$). This is because the algorithm is able to prune away half of the branches at each level of the tree.

% \subsection{Sub Section 1}
% \label{sub_sec:sub_section_1}

% \begin{theorem}
% This is a theorem.
% \end{theorem}
% \begin{proof}
% This is a proof.
% \end{proof}
% \begin{example}
% This is an example.
% \end{example}
% \begin{explanation}
% This is an explanation.
% \end{explanation}
% \begin{claim}
% This is a claim.
% \end{claim}
% \begin{corollary}
% This is a corollary.
% \end{corollary}
% \begin{prop}
% This is a proposition.
% \end{prop}
% \begin{lemma}
% This is a lemma.
% \end{lemma}
% \begin{question}
% This is a question.
% \end{question}
% \begin{solution}
% This is a solution.
% \end{solution}
% \begin{exercise}
% This is an exercise.
% \end{exercise}
% \begin{definition}[Definition]
% This is a definition.
% \end{definition}
% \begin{note}
% This is a note.
% \end{note}

% subsection sub_section_1 (end)

\newpage