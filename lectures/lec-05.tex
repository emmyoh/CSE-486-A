\lesson{Wed, 13 September 2023, 11:40am -- 1:00pm}{Week 3, Wednesday}

\begin{definition}
    \textbf{Multiagent environments} are environments in which multiple agents share the same environment.
\end{definition}

Contingency plans are necessary to account for the unpredictability of other agents.

\begin{definition}
    Each agent has its own \textbf{utility function} that maps states to real numbers.
\end{definition}

\begin{definition}
    A \textbf{zero-sum game} is a game in which the sum of the utilities of all agents is zero. A \textbf{game} is a decision-making problem, which is a multiagent environment in which the agents' goals are in conflict. A \textbf{competitive game} is a game in which the agents' utility functions are maximised by different states.
\end{definition}

\begin{definition}
    A \textbf{game} is defined by:

    \begin{itemize}
        \item $S_{0}$, the initial state.
        \item $\text{PLAYER}(s)$, which returns the player whose turn it is in state $s$.
        \item $\text{ACTIONS}(s)$, which returns the set of legal moves in state $s$.
        \item $\text{RESULT}(s, a)$, which returns the state resulting from playing action $a$ in state $s$.
        \item $\text{TERMINAL-TEST}(s)$, which returns if $s$ is in its terminal state.
        \item $\text{UTILITY}(s, p)$ (referred to as the objective function or payoff function), which returns the final numeric value for a game that ends in terminal state $s$ for a player $p$.
    \end{itemize}
\end{definition}

\subsection{$\alpha$-$\beta$ pruning}
\label{sub_sec:alpha_beta_pruning}

\begin{definition}
    \textbf{Minimax} is a decision rule for minimizing the possible loss for a worst case (maximum loss) scenario. It is a recursive algorithm for choosing the next move in an $n$-player game, usually a two-player game. A value is associated with each position or state of the game. This value is computed by means of a \textbf{position evaluation function} and it indicates how good it would be for a player to reach that position. The player then makes the move that maximizes the minimum value of the position resulting from the opponent's possible following moves. If it is $A$'s turn to move, $A$ gives a value to each of their legal moves. $A$ will choose the move with the maximum value of the minimum values resulting from their opponent's possible following moves. If it is $B$'s turn to move, $B$ gives a value to each of their legal moves. $B$ will choose the move with the minimum value of the maximum values resulting from their opponent's possible following moves.
\end{definition}

Searching a complete tree takes $O(b^{m})$ time, where $b$ is the branching factor and $m$ is the maximum depth of the tree. This is too slow for most games. We can prune the tree to reduce the number of nodes that need to be explored.

\begin{definition}
    \textbf{Pruning} is the process of removing parts of a tree that are not relevant to the computation.\ \textbf{$\alpha$-$\beta$ pruning} is a search algorithm that seeks to decrease the number of nodes that are evaluated by the minimax algorithm in its search tree. It stops evaluating a move when at least one possibility has been found that proves the move to be worse than a previously examined move. Such moves need not be evaluated further. When applied to a standard minimax tree, it returns the same move as minimax would, but prunes away branches that cannot possibly influence the final decision.
\end{definition}

% \subsection{Sub Section 1}
% \label{sub_sec:sub_section_1}

% \begin{theorem}
% This is a theorem.
% \end{theorem}
% \begin{proof}
% This is a proof.
% \end{proof}
% \begin{example}
% This is an example.
% \end{example}
% \begin{explanation}
% This is an explanation.
% \end{explanation}
% \begin{claim}
% This is a claim.
% \end{claim}
% \begin{corollary}
% This is a corollary.
% \end{corollary}
% \begin{prop}
% This is a proposition.
% \end{prop}
% \begin{lemma}
% This is a lemma.
% \end{lemma}
% \begin{question}
% This is a question.
% \end{question}
% \begin{solution}
% This is a solution.
% \end{solution}
% \begin{exercise}
% This is an exercise.
% \end{exercise}
% \begin{definition}[Definition]
% This is a definition.
% \end{definition}
% \begin{note}
% This is a note.
% \end{note}

% subsection sub_section_1 (end)

\newpage