\lesson{Wed, 11 October 2023, 11:40am -- 1:00pm}{Week 7, Wednesday}

\subsection{Min-conflicts algorithm}
\label{sub:min_conflicts_algorithm}

CSPs can often be solved effectively with local search algorithms that use a complete state formulation---that is, the initial state assigns values to every variable, and then the search updates the values one variable at a time. With the eight queens puzzle, for example, one may begin by placing the eight queens in a random configuration, before then proceeding to move one randomly selected conflicting queen at a time until the puzzle is solved. To choose these new values, one may use the \textbf{min-conflicts heuristic}---when picking a new value for a given variable, one selects the value that would lead to a minimum number of conflicts with the other variables (and if there are multiple possible minimums, randomly select one).

\begin{algorithm}[H]
\caption{Min-conflicts algorithm}
\begin{algorithmic}[1]
\Procedure{min-conflicts}{$csp$, $max\_steps$}
\State $current \gets$ an initial complete assignment for $csp$
\For{$i = 1$ to $max\_steps$}
\If{$current$ is a solution for $csp$}
\State \textbf{return} $current$
\EndIf
    \State $var \gets$ a randomly chosen, conflicted variable from $csp.\text{VARIABLES}$
    \State $value \gets$ the value $v$ for $var$ that minimizes $\text{CONFLICTS}(var, v, current, csp)$
    \State $current[var] \gets value$
\EndFor
\State \textbf{return} \textsc{failure}
\EndProcedure
\end{algorithmic}
\end{algorithm}

On the $n$-queens problem, the performance of min-conflicts appears roughly constant with respect to $n$. Another advantage of linear search is its ability to adapt to changes in the problem, such as when solving scheduling problems with online data.


% \subsection{Sub Section 1}
% \label{sub_sec:sub_section_1}

% \begin{theorem}
% This is a theorem.
% \end{theorem}
% \begin{proof}
% This is a proof.
% \end{proof}
% \begin{example}
% This is an example.
% \end{example}
% \begin{explanation}
% This is an explanation.
% \end{explanation}
% \begin{claim}
% This is a claim.
% \end{claim}
% \begin{corollary}
% This is a corollary.
% \end{corollary}
% \begin{prop}
% This is a proposition.
% \end{prop}
% \begin{lemma}
% This is a lemma.
% \end{lemma}
% \begin{question}
% This is a question.
% \end{question}
% \begin{solution}
% This is a solution.
% \end{solution}
% \begin{exercise}
% This is an exercise.
% \end{exercise}
% \begin{definition}[Definition]
% This is a definition.
% \end{definition}
% \begin{note}
% This is a note.
% \end{note}

% subsection sub_section_1 (end)

\newpage