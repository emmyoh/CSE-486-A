\lesson{Wed, 25 October 2023, 11:40am -- 1:00pm}{Week 9, Wednesday}

\section{Naive Bayes Algorithm}\label{sec:naive_bayes_algorithm}

In the context of email spam filtering, we have a training data set and a testing data set. From the training data, we can calculate $P(\text{Spam}) = \frac{|\text{Spam}|}{|\text{Spam}| + |\text{Ham}|}$, $P(\text{Ham}) = \frac{|\text{Ham}|}{|\text{Spam}| + |\text{Ham}|}$. For each $w \in \text{Spam} \cup \text{Ham}$, where $w$ is a word, we may find $P(w \mid \text{Spam}) = \frac{P(|w \in \text{Spam}|) + 1}{|\text{Spam}| + 2}$---the $+1$ \& $+2$ ensure a probability of $50\%$ when $|w \in \text{Spam}| = 0$ (ie, when the word, $w$, appears only in the `ham' set in the training data)---and $P(w \mid \text{Ham}) = \frac{P(|w \in \text{Ham}|) + 1}{|\text{Ham}| + 2}$. Within the testing data, we have several emails, and we are aiming to determine if they're spam or not. For each $\text{Email} \in \text{Testing Set}$, we have in the email a set of distinct words, $X$, where every word in $X$ also appears in the training data (and, therefore, has a probability associated with it). Where $X := \text{Email} \cap (\text{Spam} \cup \text{Ham}): P(\text{Spam} \mid X) \approx \frac{P(\text{Spam}) \underset{x \in X}{\prod} P(x \mid \text{Spam})}{P(\text{Spam}) \underset{x \in X}{\prod} P(x \mid \text{Spam}) + P(\text{Ham}) \underset{x \in X}{\prod} P(x \mid \text{Ham})}$; if $P(\text{Spam} \mid X) > 0.5$ then we classify $\text{Email}$ as `spam', otherwise we classify $\text{Email}$ as `ham'. If $P(w \mid \text{Spam}) = 0$, then $P(\text{Spam} \mid X) = 0$, and if $P(w \mid \text{Ham}) = 0$, then $P(\text{Ham} \mid X) = 1$, requiring us to have added the $+1$s \& $+2$s earlier to prevent either of these probabilities from being $0$.

% \subsection{Sub Section 1}
% \label{sub_sec:sub_section_1}

% \begin{theorem}
% This is a theorem.
% \end{theorem}
% \begin{proof}
% This is a proof.
% \end{proof}
% \begin{example}
% This is an example.
% \end{example}
% \begin{explanation}
% This is an explanation.
% \end{explanation}
% \begin{claim}
% This is a claim.
% \end{claim}
% \begin{corollary}
% This is a corollary.
% \end{corollary}
% \begin{prop}
% This is a proposition.
% \end{prop}
% \begin{lemma}
% This is a lemma.
% \end{lemma}
% \begin{question}
% This is a question.
% \end{question}
% \begin{solution}
% This is a solution.
% \end{solution}
% \begin{exercise}
% This is an exercise.
% \end{exercise}
% \begin{definition}[Definition]
% This is a definition.
% \end{definition}
% \begin{note}
% This is a note.
% \end{note}

% subsection sub_section_1 (end)

\newpage