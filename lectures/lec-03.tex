\lesson{Wed, 6 September 2023, 11:40am -- 1:00pm}{Week 2, Wednesday}

\subsection{Uniform-cost search (UCS) -- continued}
\label{sub_sec:uniform_cost_search_continued}

UCS is optimal and complete, but its time and space complexity remain exponential in the worst case ($O(b^{1 + \lfloor \frac{C^{*}}{\epsilon} \rfloor}))$, where all edge costs are at least $\epsilon$, $\epsilon > 0$, and $C^{*}$ is the cost of the optimal solution).

\subsection{Depth-first search (DFS)}
\label{sub_sec:depth_first_search}

Depth-first search expands the \emph{deepest} node first, removing elements from memory as it proceeds.

\begin{definition}
    Depth-first search performs \textbf{chronological backtracking}; when a search hits a dead end, it backs up to the level above.
\end{definition}

\begin{definition}
    DFS is not complete without a \textbf{depth bound}, $D$.
\end{definition}

DFS is not optimal or complete. It has an exponential time complexity of $O(b^{m})$ and a linear space complexity of $O(bm)$, where $M$ is the maximum length of any path in the state space, and $b$ is the branching factor.

\subsection{Iterative deepening search (IDS)}
\label{sub_sec:iterative_deepening_search}

IDS is a combination of BFS and DFS.\@ It performs a DFS with a depth bound, $D$ (typically starting at $1$), that increases with each iteration. It is complete (when there are no loops) and optimal, and has a time complexity of $O(b^{d})$ and a space complexity of $O(bd)$, where $d$ is the depth of the shallowest goal node.

\begin{definition}
    Iterative deepening search is an example of an \textbf{`anytime' algoirhtm}; it can return a valid solution to a problem even if it is interrurpted before concluding. It is expected to find better solutions as it continues running.
\end{definition}

Generally, IDS is the preferred uninformed search algorithm when the search space is large and the depth of the solution is not known.



% \subsection{Sub Section 1}
% \label{sub_sec:sub_section_1}

% \begin{theorem}
% This is a theorem.
% \end{theorem}
% \begin{proof}
% This is a proof.
% \end{proof}
% \begin{example}
% This is an example.
% \end{example}
% \begin{explanation}
% This is an explanation.
% \end{explanation}
% \begin{claim}
% This is a claim.
% \end{claim}
% \begin{corollary}
% This is a corollary.
% \end{corollary}
% \begin{prop}
% This is a proposition.
% \end{prop}
% \begin{lemma}
% This is a lemma.
% \end{lemma}
% \begin{question}
% This is a question.
% \end{question}
% \begin{solution}
% This is a solution.
% \end{solution}
% \begin{exercise}
% This is an exercise.
% \end{exercise}
% \begin{definition}[Definition]
% This is a definition.
% \end{definition}
% \begin{note}
% This is a note.
% \end{note}

% subsection sub_section_1 (end)

\newpage