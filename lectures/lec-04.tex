\lesson{Fri, 8 September 2023, 11:40am -- 1:00pm}{Week 2, Friday}

\subsection{Informed Search}
\label{sub_sec:informed_search}

\begin{definition}
    \textbf{Informed search algorithms} use problem-specific knowledge (ie, \textbf{domain knowledge}) to find solutions more efficiently than uninformed search algorithms. They use a \textbf{heuristic function} to estimate the cost of the cheapest path from a given node to a goal node. The heuristic function is denoted $h(n)$, where $n$ is a node in the search tree. $h(n) \ge 0$ for all $n$, while an $h(n)$ close to $0$ means that $n$ is close to a goal node, while an $h(n)$ that is very large means that $n$ is far from a goal node.
\end{definition}

\subsection{$A^{*}$ Search}
\label{sub_sec:a_star_search}

$A^{*}$ search is an informed search algorithm that uses a heuristic function to estimate the cost of the cheapest path from a given node to a goal node. It uses a \textbf{cost function}, $f(n)$, to estimate the cost of the cheapest path from the start node to a goal node through $n$. $f(n)$ is defined as $f(n) = g(n) + h(n)$. $g(n)$ is the cost of the path from the start node to $n$, and $h(n)$ is the heuristic function. $A^{*}$ search expands the nodes on the frontier in order of increasing $f(n)$ values (ie, the node with the lowest $f(n)$ is expanded first).

\begin{definition}
    A heuristic, $h$, is \textbf{admissible} if it never overestimates the cost of reaching the goal, ie, $h(n) \le h^{*}(n)$, where $h^{*}(n)$ is the true cost of reaching the goal from $n$. An admissible heuristic is \textbf{optimistic}. The straight-line distance ($h_{\text{SLD}}$) between two points is an admissible heuristic for the problem of finding the shortest path between them. If the heuristic is admissible, then $A^{*}$ search is optimal.
\end{definition}

The time \& space complexity of $A^{*}$ search is polynomial when $h$ satisfies $| h(n) - h^{*}(n) | = O(\log h^{*}(n))$. When $h(n) = 0$, this condition is \emph{not} satisfied, and, in effect, $A^{*}$ search becomes UCS\@.

% \subsection{Sub Section 1}
% \label{sub_sec:sub_section_1}

% \begin{theorem}
% This is a theorem.
% \end{theorem}
% \begin{proof}
% This is a proof.
% \end{proof}
% \begin{example}
% This is an example.
% \end{example}
% \begin{explanation}
% This is an explanation.
% \end{explanation}
% \begin{claim}
% This is a claim.
% \end{claim}
% \begin{corollary}
% This is a corollary.
% \end{corollary}
% \begin{prop}
% This is a proposition.
% \end{prop}
% \begin{lemma}
% This is a lemma.
% \end{lemma}
% \begin{question}
% This is a question.
% \end{question}
% \begin{solution}
% This is a solution.
% \end{solution}
% \begin{exercise}
% This is an exercise.
% \end{exercise}
% \begin{definition}[Definition]
% This is a definition.
% \end{definition}
% \begin{note}
% This is a note.
% \end{note}

% subsection sub_section_1 (end)

\newpage